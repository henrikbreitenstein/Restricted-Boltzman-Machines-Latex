\subsection{LMG Hamiltonian represented with Pauli matrices}

To represent the LMG Hamiltonian on a quantum computer we need to first write it in terms of Pauli matrices. From the quasi-spin form \ref{eq:quasiHamilt} we need to find a conversion to what Pauli gates are to be applied to which qubits. By defining $W = 0$ we have that

\begin{equation}
H = \varepsilon \op{J}{z} + \frac{1}{2}V \left ( \op{j}{+}{2}+\op{J}{-}{2} \right ) \; .
\end{equation}


Following the lecture notes of Hjort-Jensen \cite{LipkinQuasiToPauli}, we have the mappings:

\begin{align} \label{eq:mappingJ}
    \op{j}{z}{(n)} &=\frac{1}{2}\sum_{\sigma}\sigma a^\dagger_{n\sigma}a_{n\sigma} \\
    \op{j}{\pm}{(n)}&= \hc{a}{n\pm}\op{a}{n\pm} \; ,
\end{align}

and the commutation relations:

\begin{equation}
    \left [ \op{J}{+}, \op{J}{-} \right ] = 2\op{J}{z}
\end{equation}

and 


\begin{equation}
    \left [ \op{J}{z}, \op{J}{\pm} \right ] = \pm \op{J}{\pm} \; .
\end{equation}

With the ladder operators being

\begin{equation} \label{eq:ladderJ}
    \op{J}{\pm} = \op{J}{x} \pm i\op{J}{y} \; ,
\end{equation}

we then have the total spin operator

\begin{equation}
    \op{J}{}{2} = \op{J}{x}{2} + \op{J}{y}{2} + \op{J}{z}{2} = \frac{1}{2} \left \{ \op{J}{+}, \op{J}{-} \right \} + \op{J}{z}{2} \; ,
\end{equation}

where $n$ represents which particle the operator is applied to. Defining the number operators as

\begin{align}
    \op{N}{+} &= \sum-{n} \hc{a}{n+}\op{a}{n+} \\
    \op{N}{-} &= \sum-{n} \hc{a}{n-}\op{a}{n-} \\
    \op{N} &= \sum-{n\sigma} \hc{a}{n\sigma}\op{a}{n\sigma} \; ,
\end{align}

we can see that the operator $\op{J}{z}$ counts half the difference of particles in the upper and lower levels. Since the Hamiltonian preserves the number of particles and particles can only be moved in pairs, $\op{J}{z}$ can take the values:

\begin{equation}
    \op{J}{z} \in \left \{ -\frac{N}{2}, -\frac{N}{2}+1, \dots , \frac{N}{2} - 1, \frac{N}{2} \right \}
\end{equation}

Looking at the rotation operator

$$\op{R} = e^{i \phi \op{J}{z}} \; ,$$

we have that the possible eigenvalues $r$ of the signature operator would be

\begin{align}
    r = +1 \; &, \; \op{j}{z} = 2n \\
    r = +i \; &, \; \op{j}{z} = 2n + \frac{1}{2}\\
    r = -1 \; &, \; \op{j}{z} = 2n + 1\\
    r = -i \; &, \; \op{j}{z} = 2n + \frac{3}{2} \; .
\end{align}

Though since we can only move particles in pairs we only encounter the values $r = +1$ and $r= -1$. This also means that $\op{j}{z}$ can only change as follows:

$$\op{j}{z} \rightarrow \frac{1}{2}[(\op{N}{+}\pm 2n)-(\op{N}{-}\mp 2n)] \; .$$

Which can be written as

$$\op{j}{z} \rightarrow \op{J}{z} \pm 2n \; ,$$

So for each particle we a state doublet $ \left \{ (n, +1) , (n, -1) \right \} $. To map this to a quantum circuit we would need to have all the possible outcomes represented by the $\ket{0}$ and $\ket{1}$ measurements of the qubits in the circuit. One way to do this would be to have a qubit for each state $(n, \sigma)$, then $\ket{0}$ represents an unoccupied state and $\ket{1}$ an occupied state. Such a mapping would require a qubit for each possible state, which is $2N$ since we start out with filling up the lower energy level with particles. 

Another mapping, better in this case, would be to use the fact that both degeneracy of the states are never occupied at the same time. We would then use a qubit for each degeneracy, in total $\frac{N}{2}$ qubits, where $\ket{0}$ would represent $(n, +1)$ being occupied and $\ket{1}$ would be interpreted as $(n, -1)$ being occupied. This reduces the number of qubits needed by a half. 

Using the mapping to one-body operators in \ref{eq:mappingJ} we get:

\begin{equation}
H = \varepsilon\sum_{n}\op{j}{z}{(n)} + \frac{1}{2}V\left[\left(\sum_n\op{j}{+}{(n)}\right)^2+\left(\sum_n \op{j}{-}{(n)}\right)^2\right] \; .
\end{equation}

\begin{equation}
= \varepsilon\sum_{n}\op{j}{z}{(n)} + \frac{1}{2}V \sum_{n,m}\left (\op{j}{+}{(n)}\op{j}{+}{(m)}+ \op{j}{-}{(n)}\op{j}{-}{(m)}\right ) \; .
\end{equation}

Using $\op{j}{\pm} = \op{j}{x} \pm i\op{j}{y}$ (\ref{eq:ladderJ}) we get

\begin{equation}
H = \varepsilon\sum_{n}\op{j}{z}{(n)} + \frac{1}{2}V \sum_{n < m}\left (\op{j}{x}{(n)}\op{j}{x}{(m)}+ \op{j}{z}{(n)}\op{j}{z}{(m)}\right ) \; .
\end{equation}

Then by converting the quasi-spin operators with the relations

\begin{align}
    j_x^{(n)} &= \frac{1}{2} X_n \\
    j_y^{(n)} &= \frac{1}{2} Y_n\\
    j_z^{(n)} &= \frac{1}{2} Z_n \; ,
\end{align}

we get the Hamiltonian in Pauli matrix form

\begin{equation}
    H=\frac{1}{2}\epsilon\sum_{n}Z_n+\frac{1}{2}V\sum_{n < m}(X_nX_m-Y_nY_m) \; .
\end{equation}

