\section{The Ising model}

The Ising model consists of a discrete set of particles where nearest neighbors interact. We will look at the one dimensional Ising model, with the particles arranged in a line as follows:

image of particles in a line

Where the state of the whole system is denoted $\boldsymbol{\sigma}$ and the individual states are denoted  $\sigma$. Each particle can be in either spin up or spin down state which gives their respective interactive factor

\begin{equation}
  \sigma \in \{+1,-1\} \; .
\end{equation}

We define the interaction strength between particles as a constant $J$. We sum each neighbor pair's contribution and get the total energy for that specific configuration, which then is the local energy of that configuration:

\begin{equation}
  H(\boldsymbol{\sigma}) =J\sum_{\langle i,j\rangle} \sigma_i\sigma_j \; .
  \label{eq:Ising_hamiltonian}
\end{equation}


