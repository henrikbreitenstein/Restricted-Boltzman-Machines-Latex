\section{Basic Principles}
\subsection{Wavefunction and superposition}

A quantum state is described with the help of a wavefunction, which for one dimension we write as

$$\psi(x): \mathbb{R}\rightarrow\mathbb{C} \; .$$

In and of itself the wavefunction does not have a good physical interpretation, but the squared absolute value becomes a probability distribution, such that

\begin{equation}
    P(x) = |\psi(x)|^2 \; ,
\end{equation}

meaning that $|\psi(x)|^2$ is the probability of finding the particle at position $x$. The wavefunction therefor needs to be normalized, such that 

\begin{equation}
    \int_{-\inf}^{\inf}\mathrm{d}x \psi^*\psi = 1 \; .
\end{equation}

It then also means the particle is in a undetermined position before measurement. The quantum state is a combination of all the possible positions it can be in, weighted by $\psi(x)$, in intervals for a continuum of eigenstates $\ket{\phi}$ 

\begin{equation}
    \ket{\psi} = \int\mathrm{d}x \psi(x)\ket{\phi} \; .
\end{equation}

And for a discrete set of eigenstates

\begin{equation}
    \ket{\psi} = \sum_i \psi(x_i)\ket{\phi} \; .
\end{equation}

Such a state $\ket{\psi}$ is called a superposition.

\subsection{Operators and the Schrödinger equation}

Affecting the wavefunction is done mathematically through operators. An operator maps a state to another, transforming the wavefunction. For a generalized operator $\hat{O}$ and the state $\ket{\psi}$ we have

\begin{equation}
    \hat{O}\ket{\psi} = \ket{\psi'} \; ,
\end{equation}

where $\ket{\psi'}$ is a new state. When an operator acts on a eigenstate:

\begin{equation}
    \hat{O}\ket{\phi} = \varepsilon\ket{\phi} \; ,
\end{equation}

the eigenvalue $\varepsilon$ is a quantity of what $\hat{O}$ represents. For this quantity to be something physically measurable, the operator needs to be hermitian 

\begin{equation}
    \hat{O} = \hc{O} \; ,
\end{equation}

such that $\varepsilon \in \mathbb{R}$. An important observable operator is the hamiltonian, representing the energy of the system. The hamiltonian maps a state to its energy distribution, indicating its time evolution which is dictated by the Schrödinger equation:

\begin{equation}
    i\hbar\frac{\partial}{\partial t} |\Psi(t)\rangle = \hat{H} |\Psi(t)\rangle \; ,
\end{equation}

where $t$ is time. Eigenstates of the hamiltonian are the stable energy states of the system

\begin{equation}
    H\ket{\psi_n} = E_n\ket{\psi_n} \; ,
\end{equation}

where $E_n$ is the energy of the state.

\subsection{Global and local energy}

The global energy is the energy of the whole system, the possible are values of energy that can be measured. This is the energy eigenvalues of the hamiltonian:

\begin{equation}
    E_n = \frac{\bra{\psi_n}H\ket{\psi_n}}{\braket{\psi_n}{\psi_n}} \; .
\end{equation}

The local energy is, however, the energy associated with any particular basis state:

\begin{equation}
    E_{\text{loc}}(\phi) = \frac{\bra{\phi}H\ket{\psi}}{\braket{\phi}{\psi}} \; ,
\end{equation}

where then $\phi$ is the basis state, a specific point in the space of the system, and $\psi$ is the state of the system.

\subsection{Pauli Exlusion Principle}

The Pauli Exclusion Principle says that, within a quantum system, identical fermions are limited to one per quantum state. This is expressed mathematically with a anti-symmetric wavefunction:

\begin{equation}
    \Psi(\dots,x_i,\dots,x_j,\dots) =-\Psi(\dots,x_j,\dots,x_i,\dots) \; ,
\end{equation}

where $x_i$ and $x_j$ is two particles being switched.