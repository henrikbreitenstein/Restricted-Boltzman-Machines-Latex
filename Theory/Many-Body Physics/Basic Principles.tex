\section{Basic Principles}
\subsection{Slater Determinants}

Given a system with 2 fermions. The first particle particle has quantum numbers $i$ and spin value $\alpha$, while the second particle has quantum numbers $k$ and spin value $\beta$. Since both particles are indistinguishable form each other, we have no way of knowing which particle is where. As a result the wavefunction needs to reflect this by being indifferent to particle permutation. A naive product wavefunction of the two fermions would look like

\begin{equation}
    \ket{\psi\left ( x_1, x_2 \right )} = \varphi_{i\alpha} (x_1)\varphi_{k\beta}(x_2) \; .
\end{equation}

If we try to switch the positions of $x_1$ and $x_2$ we get that

\begin{equation}
    \ket{\psi\left ( x_2, x_1 \right )} = \varphi_{i\alpha} (x_2)\varphi_{k\beta}(x_1) \; ,
\end{equation}

which fails the anti-symmetric requirement for fermionic wavefunctions, which makes them distinguishable from each other. Instead we would want a wave function accounts for both scenarios in the first place

\begin{equation}
    \ket{\psi\left ( \mathbf{x}_1, \mathbf{x}_2 \right )} = \frac{1}{\sqrt{2}} \left ( \varphi_{i\alpha} (x_1)\varphi_{k\beta}(x_2) - \varphi_{i\alpha} (x_2)\varphi_{k\beta}(x_1)\right ) \; ,
\end{equation}

where the wavefunction is a normalized combination of the two product wavefunctions. Here we have the antisymmetry

\begin{equation}
    \ket{\psi\left ( \mathbf{x}_1, \mathbf{x}_2 \right )} = - \ket{\psi\left ( x_2, x_1 \right )} \; .
\end{equation}

Which can be further generalized for a system with $N$ fermions. 

\begin{equation}\label{eq:slatdif}
    \psi(\mathbf{x}_1, \mathbf{x}_2, \ldots, \mathbf{x}_N)=\dfrac{1}{\sqrt{N!}} \left| \begin{matrix} \varphi_1(\mathbf{x}_1) & \varphi_2(\mathbf{x}_1) & \cdots & \varphi_N(\mathbf{x}_1) \\ \varphi_1(\mathbf{x}_2) & \varphi_2(\mathbf{x}_2) & \cdots & \varphi_N(\mathbf{x}_2) \\ \vdots & \vdots & \ddots & \vdots \\ \varphi_1(\mathbf{x}_N) & \varphi_2(\mathbf{x}_N) & \cdots & \varphi_N(\mathbf{x}_N) \end{matrix} \right| \; .
\end{equation}


where it gets its name from.